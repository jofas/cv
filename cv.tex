\documentclass[10pt]{article}
\usepackage{amsmath}
\usepackage{amssymb}
\usepackage[colorlinks=true, linkcolor=black]{hyperref} % Links
\usepackage{makeidx} % Indexierung
\usepackage{siunitx}
\usepackage[ngerman]{babel} % deutsche Sonderzeichen
\usepackage[utf8]{inputenc}
\usepackage{geometry} % Dokumentendesign wie Seiten- oder Zeilenabstand bestimmen
\usepackage[toc,page]{appendix}

% Graphiken
\usepackage{tikz}
\usepackage{pgfplots}
\usepackage{pgfcore}
\usepackage{pgfopts}
\usepackage{pgfornament}
\usepackage{pgf}
\usepackage{ifthen}
\usepackage{booktabs}

% Tabellen
\usepackage{tabu}
\usepackage{longtable}
\usepackage{colortbl} % Tabellen faerben
\usepackage{multirow}
\usepackage{diagbox} % Tabellenzelle diagonal splitten

\usepackage{xcolor} % Farben
\usepackage[framemethod=tikz]{mdframed} % Hintergrunderstellung
\usepackage{enumitem} % Enumerate mit Buchstaben nummerierbar machen
\usepackage{pdfpages}
\usepackage{listings} % Source-Code darstellen
\usepackage{eurosym} % Eurosymbol
\usepackage[square,numbers]{natbib}
\usepackage{here} % figure an richtiger Stelle positionieren
\usepackage{verbatim} % Blockkommentare mit \begin{comment}...\end{comment}
\usepackage{ulem} % \sout{} (durchgestrichener Text)

\usepackage{fontawesome}
\usepackage{titlesec}

% BibLaTex
\bibliographystyle{acm}

% Aendern des Anhangnamens (Seite und Inhaltsverzeichnis)
\renewcommand\appendixtocname{Anhang}
\renewcommand\appendixpagename{Anhang}

% mdframed Style
\mdfdefinestyle{codebox}{
	linewidth=2.5pt,
	linecolor=codebordercolor,
	backgroundcolor=codecolor,
	shadow=true,
	shadowcolor=black!40!white,
	fontcolor=black,
	everyline=true,
}

% Seitenabstaende
\geometry{left=10mm,right=10mm,top=5mm,bottom=5mm}

% TikZ Bibliotheken
\usetikzlibrary{
    arrows,
    arrows.meta,
    decorations,
    backgrounds,
    positioning,
    fit,
    petri,
    shadows,
    datavisualization.formats.functions,
    calc,
    shapes,
    shapes.multipart
}

\pgfplotsset{width=7cm,compat=1.15}

\definecolor{codecolor}{HTML}{EEEEEE}
\definecolor{codebordercolor}{HTML}{CCCCCC}

% Standardeinstellungen fuer Source-Code
\lstset{
    language=C,
    breaklines=true,
    keepspaces=true,
    keywordstyle=\bfseries\color{green!70!black},
    basicstyle=\ttfamily\color{black},
    commentstyle=\itshape\color{purple},
    identifierstyle=\color{blue},
    stringstyle=\color{orange},
    showstringspaces=false,
    rulecolor=\color{black},
    tabsize=2,
    escapeinside={\%*}{*\%},
}

%\input{libuml}
%\input{liberm}

\pagenumbering{gobble}

\def\sec#1{
  ~\\
  \textbf{\large{#1}} \\
  \tikz[baseline=-3pt]{
    \draw[very thick] (0,0) -- (\linewidth,0);
  }
}

\setlength\parindent{0pt}

\titleformat{\section}{\large\bf\scshape\raggedright}{}{0em}{}
  [\titlerule]

\begin{document}

\begin{center}
  \textbf{\Large{Jonas Fassbender}} \\
\end{center}
\noindent
\begin{tabu} to \linewidth {lXl}
\faMapMarker \ Cologne, Germany
  &&\faHome \ \href{https://fassbender.dev}{fassbender.dev}
  \\
\faPhone \ +49 1578 8286049
  && \faGithub \ \href{https://github.com/jofas}{jofas}
  \\
\faEnvelope \ \href{mailto:jonas@fassbender.dev}{jonas@fassbender.dev}
  && \faGitlab \ \href{https://gitlab.com/jofas}{jofas}
  \\
\end{tabu}

\section*{Synopsis}

Software Engineer with a background in Data Science and Machine
Learning, focused on predicting with certainty.
Well versed in parallel and distributed computing techniques and
patterns.
Into cloud computing from a DevOps perspective (making Ops as simple
as possible so developers like myself can focus on creating better
software).

\section*{Professional Experience}

\begin{itemize}[label={}, leftmargin=*]

\item \textbf{Jan 2020 -- now: Self-employed Software Engineer.} Main
  projects are:

  \begin{itemize}[label=\bullet]
    \item carpolice.de: API-first microservice app with a web client
      written in Flutter. Sole maintainer of 70k+ LoC

    \item Spoho

    \item CP project german bank
  \end{itemize}

\item \textbf{Aug 2018 -- Jul 2019: Working Student, RLE International.}
In a team with other students, tried finding ways for RLE
International to adopt Machine Learning as an emerging technology and
create ML-powered products and solutions for its customers.

\item \textbf{Jul 2015 -- Aug 2016: Small Business System Administrator,
Lieb EDV Beratung.}
Set up and maintained backup systems and performed general
administration tasks on Windows servers and domains for several small
businesses.
Established new office locations for customers including the setup of
hardware, phones and printers and the local network.
Fixed broken hardware and virus infected computers.
Helped securing networks against ransomware attacks.

\end{itemize}

%\begin{tabu} to \linewidth {lX}
%2020 - now &Independent Software Engineer. Mainly working on a modern
%            insurtech platform/insurance broker focused on vehicle
%            insurance for
%            \href{https://carpolice.de}{carpolice.de}. \\
%           & \\
%2018 - 2019 & Data Scientist and Programmer,
%             RLE International. Mostly Image, Text
%             Recognition and Data Sanitation tasks. We also
%             worked within the domain of Computer Graphics
%             (mesh-based CAD formats and parsing tools). \\
%           & \\
%
%2015 - 2016 & Small Business System Administrator, Lieb EDV
%              Beratung.
%\end{tabu}

\section*{Open Source Contributions}

\begin{itemize}[label={}, leftmargin=*]

\item \href{https://crates.io/users/jofas}{\textbf{Maintainer of various
  small utility crates part of the Rust ecosystem:}} Libraries
containing solutions to abstractable problems applicable in other
contexts than my own projects. Mostly concerned with (\romannumeral 1)
meta-programming based abstractions (procedural and declarative
macros), (\romannumeral 2) (de)serialization, (\romannumeral 3)
the actix-web framework and (\romannumeral 4) solving utility tasks
such as parsing an environment file or logging requests across
multiple services. 15k downloads

\item \href{https://github.com/flutter/flutter}{\textbf{Flutter:}}
Contributed to the \texttt{PaginatedDataTable} widget from Flutter's
Material Design library

\item \href{https://github.com/SpiNNakerManchester}{\textbf{SpiNNaker:}}
Contributed bug fixes and API enhancements to the Graph and Common
Python Frontends of the SpiNNaker software stack

\end{itemize}

\section*{Education}

\begin{itemize}[label={}, leftmargin=*]

\item \textbf{Sep 2019 - Sep 2020: MSc High Performance Computing with Data
  Science, University of Edinburgh.} Thesis: Deep Learning on SpiNNaker

\item \textbf{Oct 2016 - Aug 2019: BSc Computer Science, Technical
  University of Cologne.} Thesis: Approximating the Optimal Threshold
  for an Abstaining Classifier based on a Reward Function with Regression

\end{itemize}

\section*{Technologies}

\begin{tabu} to \linewidth {lX}
Programming languages
& Rust, Dart, Python, JavaScript, Julia, Bash, C, Go, Fortran \\
& \\
Cloud Computing
& Kubernetes, Docker, Google Cloud Platform, Elastic Cloud on
  Kubernetes, ROOK/Ceph, Firebase \\
& \\
Distributed and parallel programming
& MPI, OpenMP, POSIX Threads, RabbitMQ, Apache Kafka, tokio-rs,
  rayon-rs \\
& \\
Others
& Flutter, HTML+CSS, OpenID Connect, Keycloak, Git, Node.js, {\LaTeX},
  OpenSUSE/Linux, SQL \\
\end{tabu}

\end{document}
