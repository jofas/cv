\documentclass[10pt]{article}
\usepackage{amsmath}
\usepackage{amssymb}
\usepackage[colorlinks=true, linkcolor=black]{hyperref} % Links
\usepackage{makeidx} % Indexierung
\usepackage{siunitx}
\usepackage[ngerman]{babel} % deutsche Sonderzeichen
\usepackage[utf8]{inputenc}
\usepackage{geometry} % Dokumentendesign wie Seiten- oder Zeilenabstand bestimmen
\usepackage[toc,page]{appendix}

% Graphiken
\usepackage{tikz}
\usepackage{pgfplots}
\usepackage{pgfcore}
\usepackage{pgfopts}
\usepackage{pgfornament}
\usepackage{pgf}
\usepackage{ifthen}
\usepackage{booktabs}

% Tabellen
\usepackage{tabu}
\usepackage{longtable}
\usepackage{colortbl} % Tabellen faerben
\usepackage{multirow}
\usepackage{diagbox} % Tabellenzelle diagonal splitten

\usepackage{xcolor} % Farben
\usepackage[framemethod=tikz]{mdframed} % Hintergrunderstellung
\usepackage{enumitem} % Enumerate mit Buchstaben nummerierbar machen
\usepackage{pdfpages}
\usepackage{listings} % Source-Code darstellen
\usepackage{eurosym} % Eurosymbol
\usepackage[square,numbers]{natbib}
\usepackage{here} % figure an richtiger Stelle positionieren
\usepackage{verbatim} % Blockkommentare mit \begin{comment}...\end{comment}
\usepackage{ulem} % \sout{} (durchgestrichener Text)

\usepackage{fontawesome}

% BibLaTex
\bibliographystyle{acm}

% Aendern des Anhangnamens (Seite und Inhaltsverzeichnis)
\renewcommand\appendixtocname{Anhang}
\renewcommand\appendixpagename{Anhang}

% mdframed Style
\mdfdefinestyle{codebox}{
	linewidth=2.5pt,
	linecolor=codebordercolor,
	backgroundcolor=codecolor,
	shadow=true,
	shadowcolor=black!40!white,
	fontcolor=black,
	everyline=true,
}

% Seitenabstaende
\geometry{left=20mm,right=20mm,top=20mm,bottom=20mm}

% TikZ Bibliotheken
\usetikzlibrary{
    arrows,
    arrows.meta,
    decorations,
    backgrounds,
    positioning,
    fit,
    petri,
    shadows,
    datavisualization.formats.functions,
    calc,
    shapes,
    shapes.multipart
}

\pgfplotsset{width=7cm,compat=1.15}

\definecolor{codecolor}{HTML}{EEEEEE}
\definecolor{codebordercolor}{HTML}{CCCCCC}

% Standardeinstellungen fuer Source-Code
\lstset{
    language=C,
    breaklines=true,
    keepspaces=true,
    keywordstyle=\bfseries\color{green!70!black},
    basicstyle=\ttfamily\color{black},
    commentstyle=\itshape\color{purple},
    identifierstyle=\color{blue},
    stringstyle=\color{orange},
    showstringspaces=false,
    rulecolor=\color{black},
    tabsize=2,
    escapeinside={\%*}{*\%},
}

%\input{libuml}
%\input{liberm}

\pagenumbering{gobble}

\def\sec#1{
  \noindent
  \textcolor{white}{bad workaround}\\
  \textbf{#1} \\
  \tikz[baseline=-8pt]{
    \draw[very thick] (0,0) -- (\linewidth,0);
  }
  \noindent
}

\begin{document}

\begin{center}
  \textbf{\Large{Jonas Fa{\ss}bender}} \\
\end{center}
\noindent
\begin{tabu} to \linewidth {lXl}
\faMapMarker \ Cologne, Germany
  && \\
\faPhone \ +49 1578 8286049
  &&\faHome \ \href{https://fassbender.dev}{fassbender.dev}
  \\
\faEnvelope \ \href{mailto:jonas@fassbender.dev}
              {jonas@fassbender.dev}
  && \faGithub \ \href{https://github.com/jofas}
                 {github.com/jofas}
\end{tabu}

\sec{Research Interests}
\begin{tabu} to \linewidth {X}
I am interested in Machine Learning, especially
focused on Conformal Prediction, predicting with
certainty and how to implement Machine Learning models
efficiently.
Passionate about high performance computing and high availability
architectures.

\end{tabu}

\sec{Professional Experience}
\begin{tabu} to \linewidth {lX}
2020 - now &Independent Software Engineer. Mainly working on a modern
            insurtech platform/insurance broker focused on vehicle
            insurance for
            \href{https://carpolice.de}{carpolice.de}. \\
           & \\
2018 - 2019 & Data Scientist and Programmer,
             RLE International. Mostly Image, Text
             Recognition and Data Sanitation tasks. We also
             worked within the domain of Computer Graphics
             (mesh-based CAD formats and parsing tools). \\
           & \\

2015 - 2016 & Small Business System Administrator, Lieb EDV
              Beratung. Main focus were Backup Systems and
              Windows Server administration for several
              small businesses.
\end{tabu}

\sec{Education}
\begin{tabu} to \linewidth {lX}
2019 - 2020 & MSc High Performance Computing with Data Science, University of Edinburgh. \\ & \\
            & Thesis: \\
            & Deep Learning on SpiNNaker \\ & \\
            & Modules include: \\
            & Probabilistic Modeling and Reasoning,
              Advanced Message Passing Programming,
              Data Analytics with High Performance
              Computing and Extreme Computing \\ & \\
2016 - 2019 & BSc Computer Science, Technical
              University of Cologne \\ & \\
            & Thesis: \\
            & Approximating the Optimal Threshold
              for an Abstaining Classifier based
              on a Reward Function with Regression \\ & \\
            & Modules include: \\
            & Algorithms, Artificial Intelligence, Discrete
              Mathematics/Cryptography, Distributed
              Systems, Software Engineering and Theoretical
              Computer Science \\
\end{tabu}

\sec{Technologies}
\begin{tabu} to \linewidth {lX}
Programming languages & Julia, Python, Rust, Dart, Fortran, C,
                        JavaScript, Go, Bash, Java \\
                      & \\

Machine Learning libraries and frameworks & scikit-learn,
                                            Keras,
                                            Tensorflow,
                                            OpenAI Gym \\
                                          & \\

Distributed and parallel programming & MPI, OpenMP,
                                       POSIX Threads,
                                       RabbitMQ,
                                       Apache Kafka,
                                       tokio-rs \\ & \\

Visualization and graphics & Flutter, HTML, CSS, tikz, Matplotlib,
                             Unity3D, WebGL2, OpenGL 3.0\\
                           & \\

Others & {\LaTeX}, Kubernetes, Docker, Git, Numpy, Node.js,
         OpenSUSE (Linux), SQL, UML
\end{tabu}

\end{document}
