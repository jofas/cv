\documentclass[10pt]{article}
\usepackage{amsmath}
\usepackage{amssymb}
\usepackage[colorlinks=true, linkcolor=black]{hyperref} % Links
\usepackage{makeidx} % Indexierung
\usepackage{siunitx}
\usepackage[ngerman]{babel} % deutsche Sonderzeichen
\usepackage[utf8]{inputenc}
\usepackage{geometry} % Dokumentendesign wie Seiten- oder Zeilenabstand bestimmen
\usepackage[toc,page]{appendix}

% Graphiken
\usepackage{tikz}
\usepackage{pgfplots}
\usepackage{pgfcore}
\usepackage{pgfopts}
\usepackage{pgfornament}
\usepackage{pgf}
\usepackage{ifthen}
\usepackage{booktabs}

% Tabellen
\usepackage{tabu}
\usepackage{longtable}
\usepackage{colortbl} % Tabellen faerben
\usepackage{multirow}
\usepackage{diagbox} % Tabellenzelle diagonal splitten

\usepackage{xcolor} % Farben
\usepackage[framemethod=tikz]{mdframed} % Hintergrunderstellung
\usepackage{enumitem} % Enumerate mit Buchstaben nummerierbar machen
\usepackage{pdfpages}
\usepackage{listings} % Source-Code darstellen
\usepackage{eurosym} % Eurosymbol
\usepackage[square,numbers]{natbib}
\usepackage{here} % figure an richtiger Stelle positionieren
\usepackage{verbatim} % Blockkommentare mit \begin{comment}...\end{comment}
\usepackage{ulem} % \sout{} (durchgestrichener Text)

\usepackage{fontawesome}

% BibLaTex
\bibliographystyle{acm}

% Aendern des Anhangnamens (Seite und Inhaltsverzeichnis)
\renewcommand\appendixtocname{Anhang}
\renewcommand\appendixpagename{Anhang}

% mdframed Style
\mdfdefinestyle{codebox}{
	linewidth=2.5pt,
	linecolor=codebordercolor,
	backgroundcolor=codecolor,
	shadow=true,
	shadowcolor=black!40!white,
	fontcolor=black,
	everyline=true,
}

% Seitenabstaende
\geometry{left=30mm,right=30mm,top=20mm,bottom=20mm}

% TikZ Bibliotheken
\usetikzlibrary{
    arrows,
    arrows.meta,
    decorations,
    backgrounds,
    positioning,
    fit,
    petri,
    shadows,
    datavisualization.formats.functions,
    calc,
    shapes,
    shapes.multipart
}

\pgfplotsset{width=7cm,compat=1.15}

\definecolor{codecolor}{HTML}{EEEEEE}
\definecolor{codebordercolor}{HTML}{CCCCCC}

% Standardeinstellungen fuer Source-Code
\lstset{
    language=C,
    breaklines=true,
    keepspaces=true,
    keywordstyle=\bfseries\color{green!70!black},
    basicstyle=\ttfamily\color{black},
    commentstyle=\itshape\color{purple},
    identifierstyle=\color{blue},
    stringstyle=\color{orange},
    showstringspaces=false,
    rulecolor=\color{black},
    tabsize=2,
    escapeinside={\%*}{*\%},
}

%\input{libuml}
%\input{liberm}

\def\datecolsize{4cm}
\def\sec#1{
  \section*{#1}
  \tikz[baseline=-8pt]{
    \draw[very thick] (0,0) -- (\linewidth,0);
  }
  \noindent
}

\begin{document}

\begin{center}
  \textbf{\Huge{Jonas Fa{\ss}bender}} \\
\end{center}
\noindent
\begin{tabu} to \linewidth {lXl}
\faMapMarker \ Im Lerchengrund 16, 51491 Overath, Germany
  && \\
\faPhone \ +49 1578 8286049
  &&\faHome \ \href{https://fassbender.dev}{fassbender.dev}
  \\
\faEnvelope \ \href{mailto:jonas@fassbender.dev}
              {jonas@fassbender.dev}
  && \faGithub \ \href{https://github.com/jofas}
                 {github.com/jofas}
\end{tabu}

\sec{Research Interests}
\begin{tabu} to \linewidth {X}
I am currently interested in Machine Learning, especially
focused on Conformal Prediction, predicting with
certainty and how to implement Machine Learning models
efficiently.
\end{tabu}

\sec{Education}
\begin{tabu} to \linewidth {p{\datecolsize}X}
2016 - 2019 (Expected) & BSc Computer Science, Technical
                         University of Cologne. \\ & \\
                       & Relevant Projects: \\
                       &%
                        \begin{itemize}
                          \item Partial Classification
                                Forest: A Monte Carlo based
                                Meta Classifier for
                                Supervised Learning. My
                                first contact with the idea
                                of predicting with
                                certainty. Led my to
                                Conformal Prediction.
                                I intent to write
                                my Bachelor Thesis about
                                this topic.

                          \item Monte Carlo Localization:
                                Part of the Artificial
                                Intelligence course. We
                                localized a LEGO EV3 robot
                                in known surroundings using
                                a Particle Filter.

                          \item Ray Tracing Shader: I
                                ported Keenan Crane's
                                implementation\footnote{
                                  \url{https://www.cs.cmu.%
                                    edu/~kmcrane/Projects/%
                                    QuaternionJulia/paper.%
                                    pdf}}
                                of a Ray Tracing Shader
                                that renders a four
                                dimensional Julia Set to
                                Unity3D (ShaderLab)
                                and WebGL2/WebAssembly
                                (OpenGL 3.0). Both
                                implementations add
                                animations to the rendering
                                of the Julia Set.

                          \item An architecture for
                                distributed Reinforcement
                                Learning: Technologies we
                                used were Keras
                                (on Tensorflow), OpenAI Gym
                                and RabbitMQ.

                          \item A distributed system for
                                the IoT: We built a
                                peer-to-peer network of
                                sensor servers
                                communicating over
                                a custom protocol.
                                Implemented on
                                RaspberryPis, written in C.

                          \item A visualization tool for
                                graph algorithms: This tool
                                visualizes how paths are
                                chosen by Dijkstra's
                                shortest path and the A*
                                algorithm. Both are
                                implemented with a Priority
                                Queue based on a Binary
                                Heap.

                          \item A wine classifier: A Neural
                                Network that classifies a
                                wine dataset. Built with
                                Neuroph/Neuroph Studio.

                        \end{itemize} \\ & \\
                       & Modules studied include:
                         Algorithms, Artificial
                         Intelligence, Discrete
                         Mathematics/Cryptography,
                         Distributed Systems, Software
                         Engineering and Theoretical
                         Computer Science. \\ & \\

2013 - 2015 & A Levels, Herkenrath Upper School.
              Intensified courses were English and Geology.
\end{tabu}

\sec{Professional Experience}
\begin{tabu} to \linewidth {p{\datecolsize}X}
2018 - now & Data Scientist and Programmer,
             RLE International. Mostly Image, Text
             Recognition and Data Sanitation tasks. We also
             work within the domain of Computer Graphics
             (mesh-based CAD formats and parsing tools). \\
           & \\

2015 - 2016 & Small Business System Administrator, Lieb EDV
              Beratung. Main focus were Backup Systems and
              Windows Server administration for several
              small businesses.
\end{tabu}

\sec{Technologies}
\begin{tabu} to \linewidth {lX}
Programming languages & Python, Rust, JavaScript, C, C\#,
                        Go, Bash, Elm, Lua, Prolog, Java\\
                      & \\

Machine Learning libraries and frameworks & scikit-learn,
                                            Keras,
                                            OpenAI Gym,
                                            Tesseract-OCR,
                                            Unity3D's AI
                                            API,
                                            Neuroph/Neuroph
                                            Studio \\ & \\

Distributed and parallel programming & POSIX Threads,
                                       RabbitMQ,
                                       Apache Kafka,
                                       tokio-rs \\ & \\

Visualization and graphics & HTML, CSS, tikz, Matplotlib,
                             Unity3D, WebGL2, OpenGL 3.0\\
                           & \\

Others & {\LaTeX}, Git, Numpy, Node.js, OpenSUSE (Linux),
         SQL, PL/SQL, UML
\end{tabu}

\sec{Languages}
\begin{tabu} to \linewidth {X}
German (native) \\ \\
English (C1)
\end{tabu}

\sec{Memberships}
\begin{tabu} to \linewidth {X}
IEEE - Student Member \\ \\
IEEE Computational Intelligence Society - Student Member
\end{tabu}

\sec{Interests}
\begin{tabu} to \linewidth {X}
I enjoy sports, especially basketball. On my vacations I go
camping and hiking. I like to render mathematical
constructs and animations, for me they are aesthetically
very pleasing. On my website,
\href{https://fassbender.dev}{fassbender.dev},
you can see either a Lindenmayer-System or an animation
showing different four dimensional Julia Sets reduced to
three dimensions, depending on your device and browser.
I also enjoy contemporary and classic literature, fiction
and non-fiction alike. Furthermore I eat very spicy food
and grow my own chilies.
\end{tabu}

\end{document}
