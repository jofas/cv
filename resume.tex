\documentclass[10pt]{article}
\usepackage{amsmath}
\usepackage{amssymb}
\usepackage[colorlinks=true, linkcolor=black, urlcolor=black]{hyperref} % Links
\usepackage{makeidx} % Indexierung
\usepackage{siunitx}
\usepackage[ngerman]{babel} % deutsche Sonderzeichen
\usepackage[utf8]{inputenc}
\usepackage{geometry} % Dokumentendesign wie Seiten- oder Zeilenabstand bestimmen
\usepackage[toc,page]{appendix}

% Graphiken
\usepackage{tikz}
\usepackage{pgfplots}
\usepackage{pgfcore}
\usepackage{pgfopts}
\usepackage{pgfornament}
\usepackage{pgf}
\usepackage{ifthen}
\usepackage{booktabs}

% Tabellen
\usepackage{tabu}
\usepackage{longtable}
\usepackage{colortbl} % Tabellen faerben
\usepackage{multirow}
\usepackage{diagbox} % Tabellenzelle diagonal splitten

\usepackage{xcolor} % Farben
\usepackage[framemethod=tikz]{mdframed} % Hintergrunderstellung
\usepackage{enumitem} % Enumerate mit Buchstaben nummerierbar machen
\usepackage{pdfpages}
\usepackage{listings} % Source-Code darstellen
\usepackage{eurosym} % Eurosymbol
\usepackage[square,numbers]{natbib}
\usepackage{here} % figure an richtiger Stelle positionieren
\usepackage{verbatim} % Blockkommentare mit \begin{comment}...\end{comment}
\usepackage{ulem} % \sout{} (durchgestrichener Text)

\usepackage{fontawesome}
\usepackage{titlesec}

% BibLaTex
\bibliographystyle{acm}

% Aendern des Anhangnamens (Seite und Inhaltsverzeichnis)
\renewcommand\appendixtocname{Anhang}
\renewcommand\appendixpagename{Anhang}

% mdframed Style
\mdfdefinestyle{codebox}{
	linewidth=2.5pt,
	linecolor=codebordercolor,
	backgroundcolor=codecolor,
	shadow=true,
	shadowcolor=black!40!white,
	fontcolor=black,
	everyline=true,
}

% Seitenabstaende
\geometry{left=10mm,right=10mm,top=5mm,bottom=5mm}

% TikZ Bibliotheken
\usetikzlibrary{
    arrows,
    arrows.meta,
    decorations,
    backgrounds,
    positioning,
    fit,
    petri,
    shadows,
    datavisualization.formats.functions,
    calc,
    shapes,
    shapes.multipart
}

\pgfplotsset{width=7cm,compat=1.15}

\definecolor{codecolor}{HTML}{EEEEEE}
\definecolor{codebordercolor}{HTML}{CCCCCC}

% Standardeinstellungen fuer Source-Code
\lstset{
    language=C,
    breaklines=true,
    keepspaces=true,
    keywordstyle=\bfseries\color{green!70!black},
    basicstyle=\ttfamily\color{black},
    commentstyle=\itshape\color{purple},
    identifierstyle=\color{blue},
    stringstyle=\color{orange},
    showstringspaces=false,
    rulecolor=\color{black},
    tabsize=2,
    escapeinside={\%*}{*\%},
}

\pagenumbering{gobble}

\def\CLOUD{cloud}
\def\HPC{hpc}

\setlength\parindent{0pt}

\titleformat{\section}{\large\bf\scshape\raggedright}{}{0em}{}
  [\titlerule]

\begin{document}

\begin{center}
  \textbf{\Large{Jonas Fassbender}} \\
\end{center}
\noindent
\begin{tabu} to \linewidth {lXl}
\faMapMarker \ Cologne, Germany
  && \faHome \ \underline{\href{https://fassbender.dev}{fassbender.dev}}
  \\
\faPhone \ +49 1578 8286049
  && \faLinkedin \ \underline{\href{https://www.linkedin.com/in/jofas}{jofas}}
  \\
\faEnvelope \ \underline{\href{mailto:jonas@fassbender.dev}{jonas@fassbender.dev}}
  && \faGithub \ \underline{\href{https://github.com/jofas}{jofas}} /
  \faGitlab \ \underline{\href{https://gitlab.com/jofas}{jofas}}
  \\
\end{tabu}

\section*{Synopsis}

Software Engineer with a background in Data Science and Machine
Learning, focused on predicting with certainty.
Well versed in parallel and distributed computing techniques and
patterns.
Passionate about implementing distributed, data-heavy applications and
fast algorithms efficiently while maintaining good software practices.
Searching for the most obvious and simplest solution.

\section*{Professional Experience}

\begin{itemize}[label={}, leftmargin=*]

\item \textbf{Oct 2020 -- now: Self-employed Software Engineer.}

  \begin{itemize}[label=\bullet, leftmargin=*]
    \item \textbf{\underline{\href{https://carpolice.de}{carpolice.de.}}}
      Insurtech platform for car dealers, enabling them to sell
      car insurance products.
      Used by 300+ car dealers to sell 2k insurance policies since
      launch in April 2021.
      RESTful API-first microservice application written in Rust with
      a web client written in Dart/Flutter.
      Hosted on GKE.
      Used technologies include MongoDB, Redis, Elastic stack,
      Keycloak, ROOK/Ceph and OpenVPN.
      Creator and maintainer of 70k+ LoC.

    \item \textbf{German Sport University Cologne.}
      Created the technical domain specification for an
      application for teachers to conveniently generate rich semester
      plans that apply Inquiry-based Learning.
      Currently in the stage of raising funds for development.

    \item \textbf{\underline{\href{https://fassbender.dev/static/cp_for_loan_approval_prediction.pdf}{%
        Improving the consumer loan approval process of a German bank using Machine Learning.}}}
      Applied a Conformal Prediction based classifier to pre-reject
      loan requests likely to be declined, saving the fee of querying
      a credit bureau.
      Able to pre-reject 17\% of all rejected requests while
      maintaining an accuracy of 98\%.
      Currently not applied in production due to a policy shift of the
      bank in wake of the COVID-19 pandemic.
  \end{itemize}

\item \textbf{Sep 2018 -- Jul 2019: Working Student, RLE International.}
In a team with other students, explored ways for RLE
International to adopt Machine Learning as an emerging technology and
create ML-powered products and solutions for customers.
Applied different ML-models to OEM ECRs (Engineering Change Requests)
of an automobile manufacturer, predicting how much time an ECR is
going to take to be resolved.

\item \textbf{Jul 2015 -- Aug 2016: Small Business System Administrator,
Lieb EDV Beratung.}
Set up and maintained backup systems and performed general
administration tasks on Windows servers and domains for several small
businesses.

\end{itemize}

\section*{Open Source Contributions}

\begin{itemize}[label={}, leftmargin=*]

\item \underline{\href{https://crates.io/users/jofas}{\textbf{Creator and maintainer of various
  Rust crates.}}}
Libraries containing solutions concerned with (\romannumeral 1)
meta-programming based abstractions (procedural and declarative
macros), (\romannumeral 2) (de)serialization, (\romannumeral 3)
the actix-web framework and (\romannumeral 4) solving utility tasks
such as parsing an environment file or logging HTTP-requests across
multiple services. 20k+ downloads.

\item \underline{\href{https://github.com/flutter/flutter}{\textbf{Flutter.}}}
Contributed to the \texttt{PaginatedDataTable} widget from Flutter's
Material Design library.

\item \underline{\href{https://github.com/SpiNNakerManchester}{\textbf{SpiNNaker.}}}
Contributed bug fixes and API enhancements to the Graph and Common
Python Frontends of the SpiNNaker software stack.

\end{itemize}

\section*{Education}

\begin{itemize}[label={}, leftmargin=*]

\item \textbf{Sep 2019 - Sep 2020: MSc High Performance Computing with Data
  Science, University of Edinburgh.} Thesis: Deep Learning on SpiNNaker

\item \textbf{Oct 2016 - Aug 2019: BSc Computer Science, Technical
  University of Cologne.} Thesis: Approximating the Optimal Threshold
  for an Abstaining Classifier based on a Reward Function with Regression

\end{itemize}

\section*{Technologies}

\begin{tabu} to \linewidth {lX}
Programming languages
& Rust, Dart, Python, JavaScript, Julia, C, Fortran \\
&\\
Cloud Computing
& Kubernetes, Docker, Google Cloud Platform, Elastic Cloud on
  Kubernetes, ROOK/Ceph, Firebase \\
&\\
Data Science
& scikit-learn, Keras, Tensorflow, numpy, pandas, DataFrames.jl,
  Matplotlib \\
&\\
Others
& Flutter, HTML+CSS, OpenID Connect, Keycloak, Git, {\LaTeX},
  OpenSUSE/Linux, Bash, SQL, MongoDB, Redis, Elastic stack, RabbitMQ,
  Kafka, MPI, OpenMP \\
\end{tabu}

\end{document}
